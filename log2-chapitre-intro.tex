% ================================
\chapter{Remarques préliminaires}
% ================================

\section{Remarques sur les notations}

	\begin{liste}
		\item 
			Dans les notes, nous respectons les conventions du pseudo-code 
			qui ont été introduites en première année.
		\item
			En C++, une variable peut désigner directement un objet 
			ou bien être une référence à l'objet (ce dernier cas devant être
			indiqué explicitement). Il faut s'en rappeler et être prudent 
			lors de la traduction du pseudo-code vers le \ C++.
	\end{liste}

\section{Rappel concernant les paramètres}

	\begin{itemize}
		\item 
			Suivi d'une flèche vers le bas ($\downarrow $), la valeur initiale du 
			paramètre est nécessaire à l'algorithme pour fonctionner.
		\item 
			Suivi d'une flèche vers le haut ($\uparrow $), l'algorithme va donner 
			une valeur au paramètre à la fin de son déroulement.
		\item
			Suivi de la double flèche ($\updownarrow $), l'algorithme va utiliser 
			la valeur de départ mais va aussi la modifier si 
			elle change au cours de l'algorithme.
		\item
			On admet également les notations IN, OUT et IN/OUT en lieu et place des flèches.
		\item
			Si rien n'est indiqué, il faut comprendre qu'il s'agit d'un paramètre en entrée (IN)
	\end{itemize}